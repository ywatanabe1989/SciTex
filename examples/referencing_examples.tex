%% -*- coding: utf-8 -*-
%% Timestamp: "2025-05-05 21:00:00 (ywatanabe)"
%% File: "referencing_examples.tex"

% This file demonstrates how to reference figures and tables in your LaTeX documents
% It is meant for educational purposes and is not meant to be compiled directly

\section{Examples of Referencing Figures and Tables}

\subsection{Basic Figure References}

% Basic figure reference
As shown in Figure~\ref{fig:01}, the SciTex system follows a structured workflow for manuscript preparation and compilation.

% Reference to a specific panel in a figure
Figure~\ref{fig:06}A illustrates the workflow diagram of the figure processing system.

% Reference to multiple panels
As demonstrated in Figure~\ref{fig:06}B-D, the system provides several features to enhance user productivity.

% Reference with parenthetical panels
The architecture diagram (Figure~\ref{fig:02}) shows the hierarchical organization of components, while the detailed pipeline (Figure~\ref{fig:03}) explains each processing step.

% Reference at the beginning of a sentence
Figure~\ref{fig:04} presents user satisfaction data collected from a survey of 50 researchers.

\subsection{Basic Table References}

% Basic table reference
Table~\ref{tab:01} summarizes the different prompt types used in the SciTex system along with their success rates.

% Reference with additional context
As shown in Table~\ref{tab:02}, the performance metrics indicate a significant improvement over traditional methods (p < 0.01).

% Reference to specific cells
The highest satisfaction rating (92.5\%) was reported for the term checking functionality (Table~\ref{tab:01}, row 1).

% Reference at the beginning of a sentence
Table~\ref{tab:02} illustrates the performance comparison across different workloads.

\subsection{Advanced Referencing Techniques}

% Referencing multiple figures together
Figures~\ref{fig:01} and \ref{fig:02} together demonstrate the complete system architecture and workflow.

% Referencing a figure and table together
As shown in Figure~\ref{fig:05} and Table~\ref{tab:02}, there is a strong correlation between the processing time and user satisfaction.

% Creating a range of figures
The complete SciTex pipeline is illustrated across Figures~\ref{fig:01}--\ref{fig:05}.

% Referring to a figure in a different section
As we discussed earlier (see Figure~\ref{fig:01} in Section~\ref{sec:introduction}), the overall workflow significantly improves productivity.

\subsection{Reference Formatting in Different Contexts}

% In-text references
The workflow diagram (Fig.~\ref{fig:01}) and performance metrics (Tab.~\ref{tab:01}) demonstrate the effectiveness of the approach.

% Reference with descriptive text
The user satisfaction data, as presented in Figure~\ref{fig:04}, shows that 87\% of users reported improved productivity.

% Parenthetical reference
The architecture follows a modular design pattern (Fig.~\ref{fig:02}) that facilitates maintenance and extension.

% Reference with page number
For the detailed performance data, please refer to Table~\ref{tab:02} on page~\pageref{tab:02}.

\subsection{Common Pitfalls to Avoid}

% AVOID: Missing the ~ (non-breaking space)
% INCORRECT: Figure\ref{fig:01} 
% CORRECT: Figure~\ref{fig:01}

% AVOID: Inconsistent capitalization
% INCORRECT: figure~\ref{fig:01} 
% CORRECT: Figure~\ref{fig:01}

% AVOID: Incorrect reference type
% INCORRECT: Figure~\ref{tab:01} (referencing a table with fig:) 
% CORRECT: Table~\ref{tab:01}

% AVOID: Missing leading zeros
% INCORRECT: Figure~\ref{fig:1} 
% CORRECT: Figure~\ref{fig:01}

%%%% EOF