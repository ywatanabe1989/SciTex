% Example methods section demonstrating SciTex features

The SciTex system employs a structured approach to scientific manuscript preparation, combining traditional LaTeX components with modern AI-assisted tools. This section describes the methodology behind the system, including document organization, figure handling, and AI integration.

\subsection{Document Structure}

SciTex organizes document content into a modular structure, with separate files for each section. This approach offers several advantages:

\begin{itemize}
    \item Improved collaboration, allowing multiple authors to work on different sections
    \item Better version control with granular tracking of changes
    \item Enhanced readability and maintainability of the document code
    \item Simplified navigation and editing of complex documents
\end{itemize}

The document compilation process automatically combines these modular components into a cohesive final manuscript, as illustrated in Figure~\ref{fig:02}.

\subsection{Figure and Table Management}
\label{sec:figure_management}

Scientific manuscripts often include numerous figures and tables, which require careful handling. SciTex provides a structured approach to figure and table management with the following components:

\begin{enumerate}
    \item Standardized naming conventions for consistent referencing
    \item Separate storage of content and captions for improved organization
    \item Automated processing pipeline for format conversion and optimization
    \item Consistent styling and labeling across the document
\end{enumerate}

For figures, SciTex supports multiple input formats, including PowerPoint presentations, which are automatically converted to publication-ready PNG files. Figure~\ref{fig:03} shows the complete figure processing pipeline. Multi-panel figures are supported with standardized labeling and referencing conventions, as demonstrated in Figure~\ref{fig:06}.

\subsection{AI-Assisted Prompts}
\label{sec:ai_integration}

SciTex integrates with OpenAI's GPT models through a structured API, allowing for several AI-assisted functions. Table~\ref{tab:01} summarizes the different prompt types and their success rates in improving manuscript quality.

\begin{itemize}
    \item \textbf{Text Revision}: Improving grammar, clarity, and scientific style
    \item \textbf{Terminology Checking}: Ensuring consistent use of technical terms
    \item \textbf{Citation Insertion}: Suggesting and inserting appropriate citations
    \item \textbf{Content Expansion}: Helping develop sections from outlines
\end{itemize}

The AI integration follows an asynchronous approach, where manuscript content is sent to the GPT model for processing and then returned with suggested improvements. The author maintains full control, approving or modifying all AI-suggested changes.

\subsection{Performance Benchmarking}

To evaluate system performance, we conducted benchmark tests comparing SciTex to traditional LaTeX workflows. Figure~\ref{fig:05} presents the compilation time comparison across different manuscript sizes. As shown in the figure, SciTex achieves significantly faster compilation times, particularly for documents with numerous figures and complex formatting.

The performance metrics in Table~\ref{tab:02} provide a comprehensive comparison across different aspects of the document preparation process. These metrics were collected using standard timing and memory profiling tools during actual manuscript preparation tasks.

\subsection{Experimental Setup}

To evaluate the effectiveness of the SciTex system, we conducted a user study with 25 researchers from diverse scientific disciplines. Participants were asked to prepare a sample manuscript using both traditional methods and the SciTex system.

We collected data on:
\begin{itemize}
    \item Time spent on different aspects of manuscript preparation
    \item Number of formatting and style issues in the final document
    \item User satisfaction and perceived usefulness (Figure~\ref{fig:04})
    \item Learning curve and adoption challenges
\end{itemize}

The experiment used a within-subjects design with counterbalanced conditions to control for order effects. Data was analyzed using paired t-tests with a significance threshold of $p < 0.05$.

\subsection{Analysis Methods}

User performance data was analyzed using standard statistical methods, including:

\begin{equation}
\label{eq:effect_size}
d = \frac{\bar{x}_1 - \bar{x}_2}{s_p}
\end{equation}

Where $d$ is Cohen's effect size, $\bar{x}_1$ and $\bar{x}_2$ are the means of the two conditions, and $s_p$ is the pooled standard deviation.

For qualitative data, we employed thematic analysis \cite{braun2006using} to identify key patterns in user feedback. This involved coding participant responses, identifying recurring themes, and synthesizing findings into a cohesive narrative.