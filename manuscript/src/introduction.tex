%% -*- coding: utf-8 -*-
%% Timestamp: "2025-05-06 20:02:18 (ywatanabe)"
%% File: "/home/ywatanabe/proj/SciTex/manuscript/src/introduction.tex"

\section{Introduction}
\label{sec:introduction}

Scientific writing in LaTeX presents challenges for researchers unfamiliar with the system or managing documents with multiple figures, tables, and citations \cite{Smith2020}. SciTex addresses these issues with a modular LaTeX template designed specifically for scientific manuscripts.

This template organizes content into separate files with clear structure and provides tools for automating common document preparation tasks. SciTex creates a structured workflow for manuscript preparation that helps researchers focus on content rather than formatting.

% Note the cross-referencing capability using the \ref{} command
% This demonstrates LaTeX's cross-referencing system

Key features of the SciTex system include:

\begin{itemize}
    \item Modular organization of manuscript content into separate files
    \item Structured figure and table management system
    \item Automated compilation with version tracking
    \item Consistent formatting and referencing throughout the document
\end{itemize}

The template system builds on established scientific writing methodologies while adding practical features for modern document preparation \cite{Taylor2022}. SciTex employs a modular architecture with key components detailed in Section~\ref{sec:methods}.

% This is a demonstration of how to cite references from the bibliography.bib file
% You can use \cite{} for standard citations
% Or \citep{} and \citet{} for parenthetical and textual citations

Document preparation can be a time-intensive aspect of research, with formatting and citation management being particularly demanding tasks \cite{Lee2018}. By providing automation for these aspects, SciTex aims to help researchers focus more on scientific content rather than document formatting details.

In this paper, we present the design and implementation of SciTex, followed by evaluation results and usage examples.

%%%% EOF