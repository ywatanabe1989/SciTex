%% -*- coding: utf-8 -*-
%% Timestamp: "2025-05-05 12:25:00 (ywatanabe)"
%% File: "/home/ywatanabe/proj/SciTex/manuscript/src/discussion.tex"

\section{Discussion}
\label{sec:discussion}

% This section demonstrates effective discussion structure and LaTeX features

Our results demonstrate that SciTex significantly improves the efficiency and quality of scientific manuscript preparation. The observed 45\% reduction in document preparation time aligns with findings from previous studies on scientific writing tools \cite{Robinson2023}, but the additional benefits of AI-assisted revision represent a novel advancement.

% This demonstrates how to effectively discuss results and compare to previous work

\subsection{Implications for Scientific Writing}
\label{subsec:implications}

The integration of AI assistance for text revision and citation management has several important implications for scientific writing practices. First, it addresses the well-documented challenges of maintaining consistent terminology and style throughout lengthy documents \cite{Edwards2019}. Second, it reduces the cognitive load on researchers, allowing them to focus more on scientific content rather than formatting concerns.

As shown in Section~\ref{sec:results}, users particularly valued the improved manuscript organization and reduced time spent on formatting. This suggests that SciTex succeeds in its primary goal of streamlining the document preparation process while maintaining high quality standards.

% This demonstrates how to reference previous sections

\subsection{Limitations and Future Directions}
\label{subsec:limitations}

Despite its advantages, SciTex has several limitations that should be addressed in future work. First, while the AI revision capabilities are generally effective (Table~\ref{tab:quality-metrics}), they occasionally struggle with highly specialized scientific terminology or complex mathematical expressions. This limitation could be addressed through domain-specific training of language models.

Second, as noted by 32\% of users, the figure management system could be more intuitive. Future versions should incorporate a more visual interface for organizing and arranging figures, potentially through integration with graphical editors.

% This demonstrates how to create a numbered list in LaTeX

Several promising directions for future development include:

\begin{enumerate}
    \item \textbf{Expanded journal template library} - Adding support for more publisher-specific formats
    \item \textbf{Real-time collaborative editing} - Incorporating web-based concurrent editing capabilities
    \item \textbf{Smart citation recommendations} - Using AI to suggest relevant papers based on manuscript content
    \item \textbf{Integrated literature review tools} - Adding features to assist with literature synthesis and comparison
\end{enumerate}

\subsection{Broader Impact}
\label{subsec:impact}

The broader impact of tools like SciTex extends beyond individual efficiency gains. By reducing the technical barriers to effective LaTeX usage, such tools can democratize access to high-quality document preparation, particularly benefiting researchers with limited experience or resources. This aligns with calls for more accessible scientific communication tools \cite{Patel2022}.

Furthermore, standardized templates can improve the consistency and readability of scientific literature as a whole, potentially enhancing knowledge transfer and interdisciplinary collaboration.

In conclusion, SciTex represents a significant step forward in scientific document preparation by combining the structural advantages of LaTeX with the efficiency benefits of AI assistance and automation. While there is room for improvement, the current implementation already offers substantial benefits to researchers across disciplines.

%%%% EOF