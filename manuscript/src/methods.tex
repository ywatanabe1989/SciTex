%% -*- coding: utf-8 -*-
%% Timestamp: "2025-05-05 12:15:00 (ywatanabe)"
%% File: "/home/ywatanabe/proj/SciTex/manuscript/src/methods.tex"

\section{Methods}
\label{sec:methods}

% This demonstrates how to create subsections and use LaTeX formatting features

\subsection{Template Architecture}
\label{subsec:architecture}

The SciTex template is organized into a hierarchical structure for modularity and reuse. The template consists of three main components:

\begin{enumerate}
    \item \textbf{Main document entry point} - \verb|base.tex| controls the overall document structure
    \item \textbf{Content sections} - Individual .tex files in the \verb|src/| directory contain the actual content
    \item \textbf{Style definitions} - Files in \verb|src/styles/| control formatting and appearance
\end{enumerate}

This architecture allows authors to focus on content without worrying about formatting details. The separation of content and formatting makes it easier to adapt the template to different journal requirements by modifying only the style files.

% This demonstrates the use of equations in LaTeX

\subsection{Document Compilation Process}
\label{subsec:compilation}

The compilation process uses a structured approach that converts modular source files into a formatted PDF document. Let $S$ represent the source files, $C$ the compilation configuration, and $D$ the output document. The compilation process can be expressed as:

\begin{equation}
D = f_{\text{compile}}(C \oplus S)
\end{equation}

where $f_{\text{compile}}$ is the LaTeX compilation function and $\oplus$ represents the integration of configuration with source files. The configuration $C$ contains parameters for document formatting and structure.

% This demonstrates how to reference tables

\subsection{Figure and Table Management}
\label{subsec:figure-management}

Figures and tables are managed through a standardized pipeline that includes:

\begin{itemize}
    \item Automatic conversion of PowerPoint slides to TIF format (requires Windows with PowerPoint via WSL)
    \item Automated cropping to remove excess whitespace
    \item LaTeX wrapper generation for consistent formatting
    \item Directory structure for organizing source and compiled files
\end{itemize}

This standardized pipeline ensures consistent figure and table presentation throughout the document.

% This demonstrates how to create a simple table in LaTeX

\begin{table}[h!]
\centering
\caption{Components of the SciTex System}
\label{tab:components}
\begin{tabular}{lp{8cm}}
\hline
\textbf{Component} & \textbf{Description} \\
\hline
LaTeX Template & Modular document structure with sections for introduction, methods, results, etc. \\
Python Scripts & Tools for text revision, citation insertion, and terminology checking \\
Shell Scripts & Automation for compilation, figure processing, and version management \\
Documentation & Usage guides and examples for users \\
\hline
\end{tabular}
\end{table}

\subsection{Version Control and Change Tracking}
\label{subsec:version-control}

SciTex integrates with Git for version control, providing benefits such as:

\begin{itemize}
    \item Tracking changes to all document components
    \item Facilitating collaboration between multiple authors
    \item Maintaining a history of document revisions
    \item Enabling branching for experimental content
\end{itemize}

This integration is managed through shell scripts that handle common Git operations and maintain a clean version history.

\subsection{Diff Functionality for Collaborative Revision}
\label{subsec:diff-functionality}

SciTex includes an automated diff generation system that highlights changes between document versions. This feature is particularly valuable for researcher-AI collaborative work, as it:

\begin{itemize}
    \item Automatically produces a highlighted version showing additions, deletions, and modifications
    \item Enables researchers to quickly review changes suggested by AI tools
    \item Facilitates efficient editorial decisions by isolating only the modified content
    \item Creates a permanent record of collaborative changes for each version
\end{itemize}

The diff functionality uses \texttt{latexdiff} to generate a visual comparison between versions, with deleted text shown in red strikethrough formatting and added text highlighted in blue. This visual differentiation makes the collaborative revision process more transparent and efficient, allowing researchers to focus on substantive changes while maintaining full control over the document's evolution.

% This demonstrates LaTeX cross-referencing capabilities

For implementation details of these methods, please refer to the code repository and documentation. The results of applying these methods are presented in Section~\ref{sec:results}.

%%%% EOF