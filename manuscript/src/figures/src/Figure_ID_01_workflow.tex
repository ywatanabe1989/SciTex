%% -*- coding: utf-8 -*-
%% Timestamp: "2025-05-05 12:45:00 (ywatanabe)"
%% File: "/home/ywatanabe/proj/SciTex/manuscript/src/figures/src/Figure_ID_01_workflow.tex"

% This is an example figure file for the SciTex template.
% It demonstrates how to structure a figure in LaTeX.

\begin{figure}[ht!]
    \centering
    % In a real document, you would include a real image here:
    % \includegraphics[width=0.8\textwidth]{./path/to/figure.pdf}
    
    % For this example, we'll create a simple diagram using TikZ
    \begin{tikzpicture}[
        block/.style={rectangle, draw, fill=blue!20, 
                     text width=2.5cm, text centered, rounded corners, minimum height=1.5cm},
        line/.style={draw, -latex'},
        cloud/.style={draw, ellipse, fill=red!20, minimum height=1cm}
    ]
    
    % Place blocks
    \node [block] (manuscript) {Manuscript Preparation};
    \node [block, right=of manuscript] (AI) {AI-Assisted Revision};
    \node [block, right=of AI] (compile) {LaTeX Compilation};
    \node [block, below=of AI] (figures) {Figure Processing};
    \node [block, below=of manuscript] (citations) {Citation Management};
    \node [cloud, below=of compile] (output) {Final Document};
    
    % Connect blocks with arrows
    \path [line] (manuscript) -- (AI);
    \path [line] (AI) -- (compile);
    \path [line] (figures) -- (compile);
    \path [line] (citations) -- (manuscript);
    \path [line] (compile) -- (output);
    \path [line] (manuscript) to[out=250,in=110] (citations);
    \path [line] (manuscript) to[out=290,in=90] (figures);
    
    \end{tikzpicture}
    
    \caption{\textbf{SciTex workflow diagram.} The figure illustrates the key components and workflow of the SciTex system, including manuscript preparation, AI-assisted revision, figure processing, citation management, and LaTeX compilation to generate the final document. The modular design allows for customization at each stage of the process.}
    \label{fig:workflow}
\end{figure}

%%%% EOF