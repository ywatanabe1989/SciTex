%% -*- coding: utf-8 -*-
%% Timestamp: "2025-05-05 12:20:00 (ywatanabe)"
%% File: "/home/ywatanabe/proj/SciTex/manuscript/src/results.tex"

\section{Results}
\label{sec:results}

% This section demonstrates various LaTeX features and how to present results effectively

\subsection{Structure and Organization}
\label{subsec:structure}

SciTex organizes manuscript content into separate directories with clear naming conventions. This modular approach separates content from formatting, making it easier to update individual sections without affecting the entire document.

% This demonstrates how to reference sections

Key structural features include:

\begin{itemize}
    \item Content files separated by section (introduction, methods, results, etc.)
    \item Style definitions isolated in a dedicated directory
    \item Automated figure and table management with consistent labeling
    \item Version control tracking for document revisions
\end{itemize}

% This demonstrates how to use inline math and special characters in LaTeX

The template allows for mathematical content with standard LaTeX notation: $E = mc^2$, chemical formulas: H$_2$O, and special symbols: $\alpha$, $\beta$, $\Delta$. Complex equations can be typeset using the equation environment.

\subsection{System Workflow}
\label{subsec:workflow}

SciTex follows a structured workflow that processes various content types through dedicated pipelines, as illustrated in Figure~\ref{fig:01_workflow}. The workflow begins with the main entry point (\verb|base.tex|) that coordinates the compilation of all individual content files.

Each content type follows a specific processing path:

\begin{itemize}
    \item Document content files are processed through the main compilation pipeline
    \item Figures undergo specialized pre-processing including format conversion and optimization
    \item Tables are formatted and standardized for consistent presentation
    \item References are managed through a structured bibliography system
\end{itemize}

The compilation process generates both the final document (\verb|compiled.pdf|) and a diff visualization (\verb|diff.pdf|) that highlights changes between versions. This systematic approach ensures consistent outputs while maintaining traceability of changes.

\subsection{Table Features}
\label{subsec:tables}

SciTex provides structured table management with automated formatting and consistent referencing. Tables can be referenced using the standard LaTeX cross-referencing system.

% This demonstrates how to create a more complex LaTeX table with multiple columns

\begin{table}[h!]
\centering
\caption{LaTeX Template Features}
\label{tab:features}
\begin{tabular}{lp{8cm}}
\hline
\textbf{Feature} & \textbf{Description} \\
\hline
Modular Structure & Separates content into individual files for easier maintenance \\
Figure Management & Automated pipeline for consistent figure formatting and referencing \\
Table System & Standardized table creation with proper spacing and formatting \\
Bibliography & Integrated BibTeX support with flexible citation styles \\
Version Control & Built-in Git integration for tracking document changes \\
\hline
\end{tabular}
\end{table}

\subsection{Figure Management}
\label{subsec:figures}

SciTex includes a structured figure management system with the following capabilities:

\begin{itemize}
    \item Dedicated directories for source files, captions, and compiled outputs
    \item Consistent naming conventions for automatic referencing
    \item Support for multiple figure formats (TIF, JPG, PNG)
    \item Automated conversion utilities for PowerPoint slides
    \item Image optimization with automatic cropping
\end{itemize}

% This demonstrates how to create a simple bulleted list with LaTeX

The figure pipeline automates several common tasks:

\begin{itemize}
    \item Formats figures with consistent spacing and borders
    \item Places figures in appropriate sections with proper referencing
    \item Ensures high-quality image reproduction in the final document
    \item Maintains a clean directory structure for source materials
\end{itemize}

This organizational approach maintains separation between content and presentation. Additional documentation on figure management is provided in the repository documentation.

\subsection{Document Change Visualization}
\label{subsec:diff-visualization}

SciTex's diff functionality provides a practical solution for tracking and visualizing document changes. By leveraging the \texttt{latexdiff} utility, the system:

\begin{itemize}
    \item Generates side-by-side comparisons between versions with color-coded changes
    \item Tracks modifications across all document components, including equations and references
    \item Preserves the PDF format for easy sharing and annotation
    \item Maintains version history with sequential numbering (e.g., v001, v002)
\end{itemize}

Figure~\ref{fig:00_template} demonstrates the template structure. When changes are made to document content, the diff visualization shows added content in blue underlined text and deleted content in red strikethrough text, providing immediate visual indicators of modifications. This feature proves especially valuable during collaborative editing cycles, where multiple contributors may suggest changes across different document sections.

%%%% EOF