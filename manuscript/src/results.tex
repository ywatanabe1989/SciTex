%% -*- coding: utf-8 -*-
%% Timestamp: "2025-05-05 12:20:00 (ywatanabe)"
%% File: "/home/ywatanabe/proj/SciTex/manuscript/src/results.tex"

\section{Results}
\label{sec:results}

% This section demonstrates various LaTeX features and how to present results effectively

\subsection{Efficiency Improvements}
\label{subsec:efficiency}

To evaluate the efficiency gains from using SciTex, we conducted a comparative study with 50 researchers across different disciplines. Figure~\ref{fig:time-comparison} shows the average time spent on various document preparation tasks with and without SciTex.

% This demonstrates how to reference figures

The most significant improvements were observed in figure preparation (72\% reduction in time) and citation management (68\% reduction). Overall, researchers reported a 45\% reduction in total document preparation time when using SciTex compared to traditional methods.

% This demonstrates how to use inline math and special characters in LaTeX

The mean time savings ($\Delta t$) can be calculated as $\Delta t = t_{\text{traditional}} - t_{\text{SciTex}}$, which was 14.3 $\pm$ 2.7 hours per manuscript (mean $\pm$ SD, $n = 50$, $p < 0.001$).

\subsection{Quality Assessment}
\label{subsec:quality}

We evaluated the quality of AI-assisted revisions by comparing original text, AI-revised text, and expert-revised text across 30 sample paragraphs. As shown in Table~\ref{tab:quality-metrics}, the AI-revised text achieved scores comparable to expert revisions in clarity and grammar, while maintaining scientific accuracy.

% This demonstrates how to create a more complex LaTeX table with multiple columns

\begin{table}[h!]
\centering
\caption{Quality Assessment Metrics (Scale 1-10)}
\label{tab:quality-metrics}
\begin{tabular}{lccc}
\hline
\textbf{Metric} & \textbf{Original Text} & \textbf{AI-Revised} & \textbf{Expert-Revised} \\
\hline
Grammar         & $6.2 \pm 1.3$ & $8.7 \pm 0.8$ & $9.2 \pm 0.5$ \\
Clarity         & $5.8 \pm 1.5$ & $8.3 \pm 0.9$ & $8.9 \pm 0.6$ \\
Scientific Accuracy & $8.5 \pm 0.7$ & $8.4 \pm 0.8$ & $8.9 \pm 0.3$ \\
Style Consistency  & $6.7 \pm 1.2$ & $8.5 \pm 0.7$ & $8.8 \pm 0.4$ \\
Overall Quality    & $6.8 \pm 1.1$ & $8.5 \pm 0.6$ & $9.0 \pm 0.3$ \\
\hline
\end{tabular}
\end{table}

\subsection{User Experience}
\label{subsec:user-experience}

Feedback from users (n=50) indicated high satisfaction with SciTex, with 92\% of participants reporting that they would use it for future manuscripts. Key advantages cited by users included:

\begin{itemize}
    \item Reduced time spent on formatting (cited by 94\% of users)
    \item Improved manuscript organization (cited by 89\%)
    \item Helpful AI suggestions for citations and revisions (cited by 87\%)
    \item Easier collaboration with co-authors (cited by 76\%)
\end{itemize}

Figure~\ref{fig:user-satisfaction} shows the distribution of satisfaction scores across different aspects of the system.

% This demonstrates how to create a simple bulleted list with LaTeX

Areas for improvement identified by users included:

\begin{itemize}
    \item More intuitive figure management (cited by 32\% of users)
    \item Better integration with cloud platforms (cited by 28\%)
    \item Additional journal template options (cited by 23\%)
\end{itemize}

These results demonstrate that SciTex effectively addresses the common challenges in scientific manuscript preparation. The implications of these findings are discussed in Section~\ref{sec:discussion}.

%%%% EOF