%% -*- coding: utf-8 -*-
%DIF LATEXDIFF DIFFERENCE FILE
%DIF DEL ./versions/compiled_v040.tex   Wed May  7 04:35:45 2025
%DIF ADD ./compiled.tex                 Wed May  7 04:40:17 2025
%% Timestamp: "2025-05-06 23:41:14 (ywatanabe)"
%% File: "/home/ywatanabe/proj/SciTex/manuscript/base.tex"
\UseRawInputEncoding

%% ----------------------------------------
%% SETTINGS
%% ----------------------------------------
%% Columns
%% \documentclass[final,3p,times,twocolumn]{elsarticle} %% Use it for submission
%% Use the options 1p,twocolumn; 3p; 3p,twocolumn; 5p; or 5p,twocolumn
%% for a journal layout:
%% \documentclass[final,1p,times]{elsarticle}
%% \documentclass[final,1p,times,twocolumn]{elsarticle}
%% \documentclass[final,3p,times]{elsarticle}
%% \documentclass[final,3p,times,twocolumn]{elsarticle}
%% \documentclass[final,5p,times]{elsarticle}
%% \documentclass[final,5p,times,twocolumn]{elsarticle}
\documentclass[preprint,review,12pt]{elsarticle}\n
%% -*- coding: utf-8 -*-
%% Timestamp: "2025-05-06 11:52:14 (ywatanabe)"
%% File: "/home/ywatanabe/proj/SciTex/manuscript/src/styles/packages.tex"
%% preamble
\usepackage[english]{babel}
\usepackage[table]{xcolor} % For coloring tables
\usepackage{booktabs} % For professional quality tables
\usepackage{colortbl} % For coloring cells in tables
\usepackage{amsmath, amssymb} % For mathematical symbols and environments
\usepackage{amsthm} % For theorem-like environments
\usepackage{lipsum} % just for sample text
\usepackage{natbib}
\usepackage{graphicx}
\usepackage{indentfirst}
\usepackage{bashful}
\usepackage[margin=10pt,font=small,labelfont=bf,labelsep=endash]{caption}
\usepackage{calc}
\usepackage[T1]{fontenc} % [REVISED]
\usepackage[utf8]{inputenc} % [REVISED]
\usepackage{hyperref}
\usepackage{accsupp}
\usepackage{lineno}
% Tables
\usepackage{longtable}
\usepackage{supertabular}
\usepackage{tabularx}
\usepackage[pass]{geometry}
\usepackage{pdflscape}
\usepackage{csvsimple}
\usepackage{xltabular}
\usepackage{booktabs}
\usepackage{siunitx}
\usepackage{makecell}
\sisetup{round-mode=figures,round-precision=3}
\renewcommand\theadfont{\bfseries}
\renewcommand\theadalign{c}
\newcolumntype{C}[1]{>{\centering\arraybackslash}m{#1}}
\renewcommand{\arraystretch}{1.5}
\definecolor{lightgray}{gray}{0.95}

% TikZ packages for figures
\usepackage{tikz}
\usepackage{pgfplots}
\usepackage{pgfplotstable}
\usetikzlibrary{positioning,shapes,arrows,fit,calc,graphs,graphs.standard}

%% Diff
\usepackage{xcolor}
\usepackage[most]{tcolorbox} % for boxes with transparency

%% Referencing to external files
%% \usepackage{xr}
\usepackage{xr-hyper}

%%%% EOF\n
% Edit to ref between main text & supplemental material
\usepackage{xr}
\makeatletter
\newcommand*{\addFileDependency}[1]{% argument=file name and extension
  \typeout{(#1)}
  \@addtofilelist{#1}
  \IfFileExists{#1}{}{\typeout{No file #1.}}
}
\makeatother

\newcommand*{\link}[2][]{%
    \externaldocument[#1]{#2}%
    \addFileDependency{#2.tex}%
    \addFileDependency{#2.aux}%
}
\n
%% Image width
\newlength{\imagewidth}
\newlength{\imagescale}

%% Line numbers
\linespread{1.2}
\linenumbers

% Define colors with transparency (opacity value)
\definecolor{GreenBG}{rgb}{0,1,0}
\definecolor{RedBG}{rgb}{1,0,0}
% Define tcolorbox environments for highlighting
\newtcbox{\greenhighlight}[1][]{%
  on line,
  colframe=GreenBG,
  colback=GreenBG!50!white, % 50% transparent green
  boxrule=0pt,
  arc=0pt,
  boxsep=0pt,
  left=1pt,
  right=1pt,
  top=2pt,
  bottom=2pt,
  tcbox raise base
}
\newtcbox{\redhighlight}[1][]{%
  on line,
  colframe=RedBG,
  colback=RedBG!50!white, % 50% transparent red
  boxrule=0pt,
  arc=0pt,
  boxsep=0pt,
  left=1pt,
  right=1pt,
  top=2pt,
  bottom=2pt,
  tcbox raise base
}
\newcommand{\REDSTARTS}{\color{red}}
\newcommand{\REDENDS}{\color{black}}
\newcommand{\GREENSTARTS}{\color{green}}
\newcommand{\GREENENDS}{\color{black}}

% New command to read word counts
\newread\wordcount
\newcommand\readwordcount[1]{%
  \openin\wordcount=#1
  \read\wordcount to \thewordcount
  \closein\wordcount
  \thewordcount
}

\newcommand{\hl}[1]{\colorbox{yellow}{#1}}

%% Reference
\usepackage{refcount}


%% \let\oldref\ref
%% \renewcommand{\ref}[1]{%
%%   \ifnum\getrefnumber{#1}=0
%%     \sethlcolor{yellow}\hl{??}%
%%   \else
%%     \oldref{#1}%
%%   \fi
%% }

\let\oldref\ref
\newcommand{\hlref}[1]{%
  \ifnum\getrefnumber{#1}=0
    \hl{\ref*{#1}}%
    %% \sethlcolor{yellow}\hl{\ref*{#1}}%    
  \else
    \ref{#1}%
  \fi
}

% To add an 'S' prefix to a reference
\newcommand*\sref[1]{%
    S\hlref{#1}}
 
% For 'Supplementary Figure S1'
\newcommand*\sfref[1]{%
    Supplementary Figure S\hlref{#1}}
 
% For 'Supplementary Table S1'
\newcommand*\stref[1]{%
    Supplementary Table S\hlref{#1}}
 
% For 'Supplementary Materials S1'
\newcommand*\smref[1]{%
    Supplementary Materials S\hlref{#1}}


\n
\link[supple-]{../supplementary/main}

%% ----------------------------------------
%% JOURNAL NAME
%% ----------------------------------------
%% -*- coding: utf-8 -*-
%% Timestamp: "2025-05-04 08:36:10 (ywatanabe)"
%% File: "/home/ywatanabe/proj/SciTex/manuscript/src/journal_name.tex"

\journal{JOURNAL NAME HERE}

%%%% EOF\n

%% ----------------------------------------
%% START of DOCUMENT
%% ----------------------------------------
%DIF PREAMBLE EXTENSION ADDED BY LATEXDIFF
%DIF UNDERLINE PREAMBLE %DIF PREAMBLE
\RequirePackage[normalem]{ulem} %DIF PREAMBLE
\RequirePackage{color}\definecolor{RED}{rgb}{1,0,0}\definecolor{BLUE}{rgb}{0,0,1} %DIF PREAMBLE
\providecommand{\DIFaddtex}[1]{{\protect\color{blue}\uwave{#1}}} %DIF PREAMBLE
\providecommand{\DIFdeltex}[1]{{\protect\color{red}\sout{#1}}}                      %DIF PREAMBLE
%DIF SAFE PREAMBLE %DIF PREAMBLE
\providecommand{\DIFaddbegin}{} %DIF PREAMBLE
\providecommand{\DIFaddend}{} %DIF PREAMBLE
\providecommand{\DIFdelbegin}{} %DIF PREAMBLE
\providecommand{\DIFdelend}{} %DIF PREAMBLE
\providecommand{\DIFmodbegin}{} %DIF PREAMBLE
\providecommand{\DIFmodend}{} %DIF PREAMBLE
%DIF FLOATSAFE PREAMBLE %DIF PREAMBLE
\providecommand{\DIFaddFL}[1]{\DIFadd{#1}} %DIF PREAMBLE
\providecommand{\DIFdelFL}[1]{\DIFdel{#1}} %DIF PREAMBLE
\providecommand{\DIFaddbeginFL}{} %DIF PREAMBLE
\providecommand{\DIFaddendFL}{} %DIF PREAMBLE
\providecommand{\DIFdelbeginFL}{} %DIF PREAMBLE
\providecommand{\DIFdelendFL}{} %DIF PREAMBLE
%DIF HYPERREF PREAMBLE %DIF PREAMBLE
\providecommand{\DIFadd}[1]{\texorpdfstring{\DIFaddtex{#1}}{#1}} %DIF PREAMBLE
\providecommand{\DIFdel}[1]{\texorpdfstring{\DIFdeltex{#1}}{}} %DIF PREAMBLE
\newcommand{\DIFscaledelfig}{0.5}
%DIF HIGHLIGHTGRAPHICS PREAMBLE %DIF PREAMBLE
\RequirePackage{settobox} %DIF PREAMBLE
\RequirePackage{letltxmacro} %DIF PREAMBLE
\newsavebox{\DIFdelgraphicsbox} %DIF PREAMBLE
\newlength{\DIFdelgraphicswidth} %DIF PREAMBLE
\newlength{\DIFdelgraphicsheight} %DIF PREAMBLE
% store original definition of \includegraphics %DIF PREAMBLE
\LetLtxMacro{\DIFOincludegraphics}{\includegraphics} %DIF PREAMBLE
\newcommand{\DIFaddincludegraphics}[2][]{{\color{blue}\fbox{\DIFOincludegraphics[#1]{#2}}}} %DIF PREAMBLE
\newcommand{\DIFdelincludegraphics}[2][]{% %DIF PREAMBLE
\sbox{\DIFdelgraphicsbox}{\DIFOincludegraphics[#1]{#2}}% %DIF PREAMBLE
\settoboxwidth{\DIFdelgraphicswidth}{\DIFdelgraphicsbox} %DIF PREAMBLE
\settoboxtotalheight{\DIFdelgraphicsheight}{\DIFdelgraphicsbox} %DIF PREAMBLE
\scalebox{\DIFscaledelfig}{% %DIF PREAMBLE
\parbox[b]{\DIFdelgraphicswidth}{\usebox{\DIFdelgraphicsbox}\\[-\baselineskip] \rule{\DIFdelgraphicswidth}{0em}}\llap{\resizebox{\DIFdelgraphicswidth}{\DIFdelgraphicsheight}{% %DIF PREAMBLE
\setlength{\unitlength}{\DIFdelgraphicswidth}% %DIF PREAMBLE
\begin{picture}(1,1)% %DIF PREAMBLE
\thicklines\linethickness{2pt} %DIF PREAMBLE
{\color[rgb]{1,0,0}\put(0,0){\framebox(1,1){}}}% %DIF PREAMBLE
{\color[rgb]{1,0,0}\put(0,0){\line( 1,1){1}}}% %DIF PREAMBLE
{\color[rgb]{1,0,0}\put(0,1){\line(1,-1){1}}}% %DIF PREAMBLE
\end{picture}% %DIF PREAMBLE
}\hspace*{3pt}}} %DIF PREAMBLE
} %DIF PREAMBLE
\LetLtxMacro{\DIFOaddbegin}{\DIFaddbegin} %DIF PREAMBLE
\LetLtxMacro{\DIFOaddend}{\DIFaddend} %DIF PREAMBLE
\LetLtxMacro{\DIFOdelbegin}{\DIFdelbegin} %DIF PREAMBLE
\LetLtxMacro{\DIFOdelend}{\DIFdelend} %DIF PREAMBLE
\DeclareRobustCommand{\DIFaddbegin}{\DIFOaddbegin \let\includegraphics\DIFaddincludegraphics} %DIF PREAMBLE
\DeclareRobustCommand{\DIFaddend}{\DIFOaddend \let\includegraphics\DIFOincludegraphics} %DIF PREAMBLE
\DeclareRobustCommand{\DIFdelbegin}{\DIFOdelbegin \let\includegraphics\DIFdelincludegraphics} %DIF PREAMBLE
\DeclareRobustCommand{\DIFdelend}{\DIFOaddend \let\includegraphics\DIFOincludegraphics} %DIF PREAMBLE
\LetLtxMacro{\DIFOaddbeginFL}{\DIFaddbeginFL} %DIF PREAMBLE
\LetLtxMacro{\DIFOaddendFL}{\DIFaddendFL} %DIF PREAMBLE
\LetLtxMacro{\DIFOdelbeginFL}{\DIFdelbeginFL} %DIF PREAMBLE
\LetLtxMacro{\DIFOdelendFL}{\DIFdelendFL} %DIF PREAMBLE
\DeclareRobustCommand{\DIFaddbeginFL}{\DIFOaddbeginFL \let\includegraphics\DIFaddincludegraphics} %DIF PREAMBLE
\DeclareRobustCommand{\DIFaddendFL}{\DIFOaddendFL \let\includegraphics\DIFOincludegraphics} %DIF PREAMBLE
\DeclareRobustCommand{\DIFdelbeginFL}{\DIFOdelbeginFL \let\includegraphics\DIFdelincludegraphics} %DIF PREAMBLE
\DeclareRobustCommand{\DIFdelendFL}{\DIFOaddendFL \let\includegraphics\DIFOincludegraphics} %DIF PREAMBLE
%DIF LISTINGS PREAMBLE %DIF PREAMBLE
\RequirePackage{listings} %DIF PREAMBLE
\RequirePackage{color} %DIF PREAMBLE
\lstdefinelanguage{DIFcode}{ %DIF PREAMBLE
%DIF DIFCODE_UNDERLINE %DIF PREAMBLE
  moredelim=[il][\color{red}\sout]{\%DIF\ <\ }, %DIF PREAMBLE
  moredelim=[il][\color{blue}\uwave]{\%DIF\ >\ } %DIF PREAMBLE
} %DIF PREAMBLE
\lstdefinestyle{DIFverbatimstyle}{ %DIF PREAMBLE
	language=DIFcode, %DIF PREAMBLE
	basicstyle=\ttfamily, %DIF PREAMBLE
	columns=fullflexible, %DIF PREAMBLE
	keepspaces=true %DIF PREAMBLE
} %DIF PREAMBLE
\lstnewenvironment{DIFverbatim}{\lstset{style=DIFverbatimstyle}}{} %DIF PREAMBLE
\lstnewenvironment{DIFverbatim*}{\lstset{style=DIFverbatimstyle,showspaces=true}}{} %DIF PREAMBLE
%DIF END PREAMBLE EXTENSION ADDED BY LATEXDIFF

\begin{document}

%% ----------------------------------------
%% Frontmatter
%% ----------------------------------------
\begin{frontmatter}
%% -*- coding: utf-8 -*-
%% Timestamp: "2025-05-05 13:20:00 (ywatanabe)"
%% File: "/home/ywatanabe/proj/SciTex/manuscript/src/highlights.tex"

\begin{highlights}
\pdfbookmark[1]{Highlights}{highlights}

\item SciTex provides modular LaTeX templates for scientific manuscript preparation

\item The system organizes content with separate directories for manuscript sections, figures, and tables

\item Automated compilation tools simplify document generation and version tracking

\item Structured figure and table management promotes consistent formatting and referencing

\end{highlights}

%%%% EOF\n
%% -*- coding: utf-8 -*-
%% Timestamp: "2025-05-05 12:35:00 (ywatanabe)"
%% File: "/home/ywatanabe/proj/SciTex/manuscript/src/title.tex"

\title{
    SciTex: A Modular LaTeX Template System for Scientific Manuscript Preparation
}

%%%% EOF\n
%% -*- coding: utf-8 -*-
%% Timestamp: "2025-05-05 12:40:00 (ywatanabe)"
%% File: "/home/ywatanabe/proj/SciTex/manuscript/src/authors.tex"

% This demonstrates how to format author information in a scientific paper
% using the Elsevier template style

\author[1,2]{Yusuke Watanabe\corref{cor1}}
\author[3]{Claude Code}

% The \corref command marks an author as corresponding author
\corref{cor1}
\address[1]{Institute for Advanced Cocreation Studies, Osaka University, 2-2 Yamadaoka, Suita, 565-0871, Osaka, Japan}
\address[2]{NeuroEngineering Research Laboratory, Department of Biomedical Engineering, The University of Melbourne, Parkville VIC 3010, Australia}
\address[3]{Anthropic, San Francisco, CA, USA}

% The \cortext command adds the corresponding author information
\cortext[cor1]{Corresponding author. Tel: +62-XXX-XXX-XXX, Email: ywatanabe@unimelb.edu.au}

%%%% EOF\n
%% -*- coding: utf-8 -*-
%% Timestamp: "2025-05-04 08:35:38 (ywatanabe)"
%% File: "/home/ywatanabe/proj/SciTex/manuscript/src/graphical_abstract.tex"

%%Graphical abstract
%\pdfbookmark[1]{Graphical Abstract}{graphicalabstract}        
%\begin{graphicalabstract}
%\includegraphics{grabs}
%\end{graphicalabstract}

%%%% EOF\n
%% -*- coding: utf-8 -*-
%% Timestamp: "2025-05-05 12:30:00 (ywatanabe)"
%% File: "/home/ywatanabe/proj/SciTex/manuscript/src/abstract.tex"

\begin{abstract}
  \pdfbookmark[1]{Abstract}{abstract}
  % This demonstrates how to write an effective abstract for a scientific paper

  LaTeX manuscript preparation poses challenges for researchers balancing content with formatting requirements. This paper introduces SciTex, a modular LaTeX template system for scientific document preparation. SciTex organizes manuscript content into separate files with clear structure and provides automated tools for common tasks. Key features include modular document organization, automated figure and table management, version control integration, and a standardized compilation pipeline. The template includes structured directories for figures and tables with consistent naming conventions. By providing clear organization and automation of repetitive tasks, SciTex helps researchers focus on content while maintaining formatting consistency. This template serves as both a demonstration of effective LaTeX practices and a practical tool for manuscript preparation.
\end{abstract}

%%%% EOF\n
%% -*- coding: utf-8 -*-
%% Timestamp: "2025-05-05 13:15:00 (ywatanabe)"
%% File: "/home/ywatanabe/proj/SciTex/manuscript/src/keywords.tex"

\begin{keyword}
LaTeX template \sep scientific writing \sep document automation \sep version control \sep scientific manuscript
\end{keyword}

%%%% EOF\n
\end{frontmatter}

%% ----------------------------------------
%% Word Counter
%% ----------------------------------------
\begin{wordcount}
\readwordcount{./src/wordcounts/figure_count.txt} figures, \readwordcount{./src/wordcounts/table_count.txt} tables, \readwordcount{./src/wordcounts/abstract_count.txt} words for abstract, and \readwordcount{./src/wordcounts/imrd_count.txt} words for main text
\end{wordcount}

%% \begin{*wordcount}
%% \readwordcount{./src/wordcounts/figure_count.txt} figures, \readwordcount{./src/wordcounts/table_count.txt} tables, \readwordcount{./src/wordcounts/abstract_count.txt} words for abstract, and \readwordcount{./src/wordcounts/imrd_count.txt} words for main text
%% \end{*wordcount}
\n

%% ----------------------------------------
%% INTRODUCTION
%% ----------------------------------------
%% -*- coding: utf-8 -*-
%% Timestamp: "2025-05-06 20:02:18 (ywatanabe)"
%% File: "/home/ywatanabe/proj/SciTex/manuscript/src/introduction.tex"

\section{Introduction}
\label{sec:introduction}

Scientific writing in LaTeX presents challenges for researchers unfamiliar with the system or managing documents with multiple figures, tables, and citations \cite{Smith2020}. SciTex addresses these issues with a modular LaTeX template designed specifically for scientific manuscripts.

This template organizes content into separate files with clear structure and provides tools for automating common document preparation tasks. SciTex creates a structured workflow for manuscript preparation that helps researchers focus on content rather than formatting.

% Note the cross-referencing capability using the \ref{} command
% This demonstrates LaTeX's cross-referencing system

Key features of the SciTex system include:

\begin{itemize}
    \item Modular organization of manuscript content into separate files
    \item Structured figure and table management system
    \item Automated compilation with version tracking
    \item Consistent formatting and referencing throughout the document
\end{itemize}

The template system builds on established scientific writing methodologies while adding practical features for modern document preparation \cite{Taylor2022}. SciTex employs a modular architecture with key components detailed in Section~\ref{sec:methods}.

% This is a demonstration of how to cite references from the bibliography.bib file
% You can use \cite{} for standard citations
% Or \citep{} and \citet{} for parenthetical and textual citations

Document preparation can be a time-intensive aspect of research, with formatting and citation management being particularly demanding tasks \cite{Lee2018}. By providing automation for these aspects, SciTex aims to help researchers focus more on scientific content rather than document formatting details.

In this paper, we present the design and implementation of SciTex, followed by evaluation results and usage examples.

%%%% EOF\n

%% ----------------------------------------
%% METHODS
%% ----------------------------------------
%% -*- coding: utf-8 -*-
%% Timestamp: "2025-05-05 12:15:00 (ywatanabe)"
%% File: "/home/ywatanabe/proj/SciTex/manuscript/src/methods.tex"

\section{Methods}
\label{sec:methods}

% This demonstrates how to create subsections and use LaTeX formatting features

\subsection{Template Architecture}
\label{subsec:architecture}

The SciTex template is organized into a hierarchical structure for modularity and reuse. The template consists of three main components:

\begin{enumerate}
    \item \textbf{Main document entry point} - \verb|base.tex| controls the overall document structure
    \item \textbf{Content sections} - Individual .tex files in the \verb|src/| directory contain the actual content
    \item \textbf{Style definitions} - Files in \verb|src/styles/| control formatting and appearance
\end{enumerate}

This architecture allows authors to focus on content without worrying about formatting details. The separation of content and formatting makes it easier to adapt the template to different journal requirements by modifying only the style files.

% This demonstrates the use of equations in LaTeX

\subsection{Document Compilation Process}
\label{subsec:compilation}

The compilation process uses a structured approach that converts modular source files into a formatted PDF document. Let $S$ represent the source files, $C$ the compilation configuration, and $D$ the output document. The compilation process can be expressed as:

\begin{equation}
D = f_{\text{compile}}(C \oplus S)
\end{equation}

where $f_{\text{compile}}$ is the LaTeX compilation function and $\oplus$ represents the integration of configuration with source files. The configuration $C$ contains parameters for document formatting and structure.

% This demonstrates how to reference tables

\subsection{Figure and Table Management}
\label{subsec:figure-management}

Figures and tables are managed through a standardized pipeline that includes:

\begin{itemize}
    \item Automatic conversion of PowerPoint slides to TIF format (requires Windows with PowerPoint via WSL)
    \item Automated cropping to remove excess whitespace
    \item LaTeX wrapper generation for consistent formatting
    \item Directory structure for organizing source and compiled files
\end{itemize}

This standardized pipeline ensures consistent figure and table presentation throughout the document.

% This demonstrates how to create a simple table in LaTeX

\begin{table}[h!]
\centering
\caption{Components of the SciTex System}
\label{tab:components}
\begin{tabular}{lp{8cm}}
\hline
\textbf{Component} & \textbf{Description} \\
\hline
LaTeX Template & Modular document structure with sections for introduction, methods, results, etc. \\
Python Scripts & Tools for text revision, citation insertion, and terminology checking \\
Shell Scripts & Automation for compilation, figure processing, and version management \\
Documentation & Usage guides and examples for users \\
\hline
\end{tabular}
\end{table}

\subsection{Version Control and Change Tracking}
\label{subsec:version-control}

SciTex integrates with Git for version control, providing benefits such as:

\begin{itemize}
    \item Tracking changes to all document components
    \item Facilitating collaboration between multiple authors
    \item Maintaining a history of document revisions
    \item Enabling branching for experimental content
\end{itemize}

This integration is managed through shell scripts that handle common Git operations and maintain a clean version history.

\subsection{Diff Functionality for Collaborative Revision}
\label{subsec:diff-functionality}

SciTex includes an automated diff generation system that highlights changes between document versions. This feature is particularly valuable for researcher-AI collaborative work, as it:

\begin{itemize}
    \item Automatically produces a highlighted version showing additions, deletions, and modifications
    \item Enables researchers to quickly review changes suggested by AI tools
    \item Facilitates efficient editorial decisions by isolating only the modified content
    \item Creates a permanent record of collaborative changes for each version
\end{itemize}

The diff functionality uses \texttt{latexdiff} to generate a visual comparison between versions, with deleted text shown in red strikethrough formatting and added text highlighted in blue. This visual differentiation makes the collaborative revision process more transparent and efficient, allowing researchers to focus on substantive changes while maintaining full control over the document's evolution.

% This demonstrates LaTeX cross-referencing capabilities

For implementation details of these methods, please refer to the code repository and documentation. The results of applying these methods are presented in Section~\ref{sec:results}.

%%%% EOF\n

%% ----------------------------------------
%% RESULTS
 %% ----------------------------------------
%% -*- coding: utf-8 -*-
%% Timestamp: "2025-05-05 12:20:00 (ywatanabe)"
%% File: "/home/ywatanabe/proj/SciTex/manuscript/src/results.tex"

\section{Results}
\label{sec:results}

% This section demonstrates various LaTeX features and how to present results effectively

\subsection{Structure and Organization}
\label{subsec:structure}

SciTex organizes manuscript content into separate directories with clear naming conventions. This modular approach separates content from formatting, making it easier to update individual sections without affecting the entire document.

% This demonstrates how to reference sections

Key structural features include:

\begin{itemize}
    \item Content files separated by section (introduction, methods, results, etc.)
    \item Style definitions isolated in a dedicated directory
    \item Automated figure and table management with consistent labeling
    \item Version control tracking for document revisions
\end{itemize}

% This demonstrates how to use inline math and special characters in LaTeX

The template allows for mathematical content with standard LaTeX notation: $E = mc^2$, chemical formulas: H$_2$O, and special symbols: $\alpha$, $\beta$, $\Delta$. Complex equations can be typeset using the equation environment.

\subsection{System Workflow}
\label{subsec:workflow}

SciTex follows a structured workflow that processes various content types through dedicated pipelines, as illustrated in Figure~\ref{fig:01_workflow}. The workflow begins with the main entry point (\verb|base.tex|) that coordinates the compilation of all individual content files.

Each content type follows a specific processing path:

\begin{itemize}
    \item Document content files are processed through the main compilation pipeline
    \item Figures undergo specialized pre-processing including format conversion and optimization
    \item Tables are formatted and standardized for consistent presentation
    \item References are managed through a structured bibliography system
\end{itemize}

The compilation process generates both the final document (\verb|compiled.pdf|) and a diff visualization (\verb|diff.pdf|) that highlights changes between versions. This systematic approach ensures consistent outputs while maintaining traceability of changes.

\subsection{Table Features}
\label{subsec:tables}

SciTex provides structured table management with automated formatting and consistent referencing. Tables can be referenced using the standard LaTeX cross-referencing system.

% This demonstrates how to create a more complex LaTeX table with multiple columns

\begin{table}[h!]
\centering
\caption{LaTeX Template Features}
\label{tab:features}
\begin{tabular}{lp{8cm}}
\hline
\textbf{Feature} & \textbf{Description} \\
\hline
Modular Structure & Separates content into individual files for easier maintenance \\
Figure Management & Automated pipeline for consistent figure formatting and referencing \\
Table System & Standardized table creation with proper spacing and formatting \\
Bibliography & Integrated BibTeX support with flexible citation styles \\
Version Control & Built-in Git integration for tracking document changes \\
\hline
\end{tabular}
\end{table}

\subsection{Figure Management}
\label{subsec:figures}

SciTex includes a structured figure management system with the following capabilities:

\begin{itemize}
    \item Dedicated directories for source files, captions, and compiled outputs
    \item Consistent naming conventions for automatic referencing
    \item Support for multiple figure formats (TIF, JPG, PNG)
    \item Automated conversion utilities for PowerPoint slides
    \item Image optimization with automatic cropping
\end{itemize}

% This demonstrates how to create a simple bulleted list with LaTeX

The figure pipeline automates several common tasks:

\begin{itemize}
    \item Formats figures with consistent spacing and borders
    \item Places figures in appropriate sections with proper referencing
    \item Ensures high-quality image reproduction in the final document
    \item Maintains a clean directory structure for source materials
\end{itemize}

This organizational approach maintains separation between content and presentation. Additional documentation on figure management is provided in the repository documentation.

\subsection{Document Change Visualization}
\label{subsec:diff-visualization}

SciTex's diff functionality provides a practical solution for tracking and visualizing document changes. By leveraging the \texttt{latexdiff} utility, the system:

\begin{itemize}
    \item Generates side-by-side comparisons between versions with color-coded changes
    \item Tracks modifications across all document components, including equations and references
    \item Preserves the PDF format for easy sharing and annotation
    \item Maintains version history with sequential numbering (e.g., v001, v002)
\end{itemize}

Figure~\ref{fig:00_template} demonstrates the template structure. When changes are made to document content, the diff visualization shows added content in blue underlined text and deleted content in red strikethrough text, providing immediate visual indicators of modifications. This feature proves especially valuable during collaborative editing cycles, where multiple contributors may suggest changes across different document sections.

%%%% EOF\n

%% ----------------------------------------
%% DISCUSSION
%% ----------------------------------------
%% -*- coding: utf-8 -*-
%% Timestamp: "2025-05-05 12:25:00 (ywatanabe)"
%% File: "/home/ywatanabe/proj/SciTex/manuscript/src/discussion.tex"

\section{Discussion}
\label{sec:discussion}

% This section demonstrates effective discussion structure and LaTeX features

SciTex demonstrates how structured LaTeX templates can improve the organization and consistency of scientific manuscripts. The modular design provides clear separation between content and formatting, allowing for more efficient document preparation.

% This demonstrates how to effectively discuss results and compare to previous work

\subsection{Benefits of Structured Templates}
\label{subsec:benefits}

The organization of content into modular files offers several benefits for scientific writing. First, it enables focused editing of individual sections without navigating through the entire document. Second, it provides consistent formatting across the manuscript while allowing authors to concentrate on content development.

As shown in Section~\ref{sec:results}, the template includes features for figure management, table formatting, and consistent referencing. These elements help maintain document consistency and reduce manual formatting tasks.

% This demonstrates how to reference previous sections

\subsection{Areas for Customization}
\label{subsec:customization}

While the current template provides a solid foundation, users may want to customize it for specific purposes. Users can modify the template structure based on their specific requirements without disrupting the overall system.

The figure and table management systems are designed with flexibility in mind, allowing adaptation to different journal or conference formatting requirements. Users can adjust settings in the configuration files to meet specific publication guidelines.

% This demonstrates how to create a numbered list in LaTeX

Potential customizations include:

\begin{enumerate}
    \item \textbf{Journal-specific formatting} - Adjusting margins, fonts, and layout for different publishers
    \item \textbf{Citation style adaptation} - Modifying bibliography formatting for different fields or journals
    \item \textbf{Custom section organization} - Adding, removing, or reordering document sections
    \item \textbf{Figure formatting options} - Adjusting how figures are presented and captioned
\end{enumerate}

\subsection{Application Areas}
\label{subsec:applications}

The SciTex template can be applied to various document types beyond traditional research papers. With minimal adaptation, it can be used for:

\begin{itemize}
    \item Research proposals and grant applications
    \item Technical reports and white papers
    \item Conference proceedings and extended abstracts
    \item Academic theses and dissertations
\end{itemize}

The structured approach to document preparation is particularly beneficial for documents with complex elements like figures, tables, equations, and citations. The consistency in formatting across different document types helps establish a recognizable style for research groups or organizations.

\subsection{Future Work}
\label{subsec:future-work}

Several planned enhancements for the SciTex system will further improve its capabilities:

\begin{itemize}
    \item \textbf{Literature Review Assistant} - An integrated tool to help researchers organize, summarize, and cite relevant literature. This feature will:
    \begin{itemize}
        \item Import citation data from reference managers and databases
        \item Generate structured literature review templates with key information fields
        \item Provide automated citation clustering based on topic and relevance
        \item Create citation network visualizations to identify research gaps
    \end{itemize}

    \item \textbf{Mermaid Diagram Support} - Integration of Mermaid syntax for creating flowcharts, sequence diagrams, and other visualizations directly in the manuscript:
    \begin{itemize}
        \item Automatic conversion of Mermaid code to publication-quality vector graphics
        \item Template library for common scientific diagram types
        \item Visual editor for diagram creation and modification
        \item Version control for diagram evolution
    \end{itemize}

    \item \textbf{Advanced Statistical Integration} - Direct integration with statistical packages for result generation and visualization

    \item \textbf{Journal-Specific Template Repository} - A collection of pre-configured templates for major scientific journals

    \item \textbf{Collaboration Enhancement} - Real-time collaborative editing with change tracking and comment system
\end{itemize}

The literature review feature, in particular, addresses a critical pain point in scientific writing by helping researchers maintain organized connections between their work and existing literature. By structuring literature data into a standardized format, the system will facilitate evidence synthesis and gap identification, making it easier for researchers to position their work within the broader scientific context.

In conclusion, SciTex demonstrates how structured LaTeX templates can simplify the preparation of scientific documents. The modular organization, automated figure and table management, and consistent referencing system help researchers focus on content while maintaining professional document formatting. Future enhancements will continue to address the challenges researchers face in document preparation and collaboration.

%%%% EOF\n

%% ----------------------------------------
%% DATA AVAILABILITY
%% ----------------------------------------
%% -*- coding: utf-8 -*-
%% Timestamp: "2025-05-04 08:33:32 (ywatanabe)"
%% File: "/home/ywatanabe/proj/SciTex/manuscript/src/data_availability.tex"

\pdfbookmark[1]{Data Availability Statement}{data_availability}
\section*{Data Availability Statement}
Data and code used in this study is available on https://github.com/ywatanabe1989/SciTex.
\label{data and code availability}

%%%% EOF\n

%% ----------------------------------------
%% REFERENCE STYLES
%% ----------------------------------------
\pdfbookmark[1]{References}{references}
\bibliography{./src/bibliography}
% Note Re-compile is required

% %% Numbering Style (sorted and listed)
% [1, 2, 3, 4]

%% Numbering Style (sorted)
\bibliographystyle{elsarticle-num}

% Author Style
% \bibliographystyle{plainnat}
% use \citet{}

% Numbering Style (not-sorted) 
% \bibliographystyle{plainnat}
% use \cite{}


\n

%% ----------------------------------------
%% ADDITIONAL INFORMATION
%% ----------------------------------------
%% -*- coding: utf-8 -*-
%% Timestamp: "2025-05-04 08:33:31 (ywatanabe)"
%% File: "/home/ywatanabe/proj/SciTex/manuscript/src/additional_info.tex"
\pdfbookmark[1]{Additional Information}{additional_information}

\pdfbookmark[2]{Ethics Declarations}{ethics_declarations}                    
\section*{Ethics Declarations}
All study participants provided their written informed consent ...
\label{ethics declarations}

\pdfbookmark[2]{Contributors}{author_contributions}                    
\section*{Author Contributions}
Y.W. conceptualized the study, designed the template system, and developed the core components. Claude Code assisted with text refinement, content organization, and feature documentation. Y.W. supervised and validated all aspects of the final manuscript.
\label{author contributions}

\pdfbookmark[2]{Acknowledgments}{acknowledgments}                    
\section*{Acknowledgments}
This research was funded by ...
\label{acknowledgments}

\pdfbookmark[2]{Declaration of Interests}{declaration_of_interest}                    
\section*{Declaration of Interests}
The authors declare that they have no competing interests.
\label{declaration of interests}

\pdfbookmark[2]{Inclusion and Diversity Statement}{inclusion_and_diversity_statement}        
\section*{Inclusion and Diversity Statement}
We support inclusive, diverse, and equitable conduct of research.
\label{inclusion and diversity statement}

\pdfbookmark[2]{Declaration of Generative AI in Scientific Writing}{declaration_of_generative_ai}
\section*{Declaration of Generative AI in Scientific Writing}
The authors utilized Claude Code, provided by Anthropic, for enhancing manuscript content organization, text conciseness, and formatting consistency. The authors reviewed and verified all AI-generated content. Responsibility for the final manuscript content rests entirely with the authors.
\label{declaration of generative ai in scientific writing}

%% \pdfbookmark[2]{Appendices}{appendices}                    
%% \appendix
%% \section{}
%% \label{}

%%%% EOF\n

%% ----------------------------------------
%% TABLES
%% ----------------------------------------
\clearpage
\section*{Tables}
\label{tables}
\pdfbookmark[1]{Tables}{tables}
% Auto-generated file containing all table inputs
\pdfbookmark[2]{ID 01_example}{id_01_example}
\begin{table}[htbp]
\centering
\tiny
\setlength{\tabcolsep}{4pt}
\begin{tabular}{*{4}{r}}
\toprule
\textbf{\thead{$\mathrm{Parameter}$}} & \textbf{\thead{$\mathrm{Value}$}} & \textbf{\thead{$\mathrm{Unit}$}} & \textbf{\thead{$\mathrm{Notes}$}}\\
\midrule
Resolution & 300 & dpi & Standard for print\\
\rowcolor{lightgray}
Width & 2000 & pixels & Recommended for figures\\
Height & 1332 & pixels & Maintains aspect ratio\\
\rowcolor{lightgray}
File size & 48.5 & KB & After optimization\\
Compression ratio & 92.3 & \% & Size reduction achieved\\
\bottomrule
\end{tabular}
\captionsetup{width=\textwidth}
% Test table caption file
% fontsize=small
% tabcolsep=5pt
% alignment=r
% style=booktabs
% wrap-text

\caption{Test table demonstrating image optimization parameters. This table shows the key parameters used for optimizing figures in the SciTex workflow.}\n
\label{tab:01_example}
\end{table}

\restoregeometry
\n
\n

%% ----------------------------------------
%% FIGURES
%% ----------------------------------------
\clearpage
\section*{Figures}
\label{figures}
\pdfbookmark[1]{Figures}{figures}
%DIF <  Generated by compile_figure_tex_files() on Wed May  7 04:35:30 AM AEST 2025
%DIF >  Generated by compile_figure_tex_files() on Wed May  7 04:40:16 AM AEST 2025
% This file includes all figure files in order

% Figure 00: Figure 00
\begin{figure*}[p]
    \pdfbookmark[2]{Figure 00}{figure_id_00}
    \centering
    \includegraphics[width=0.95\textwidth]{./src/figures/caption_and_media/jpg/Figure_ID_00_template.jpg}
    \caption{
\textbf{
Figure 00
}
\smallskip
\\
Description for figure 00.
}
    \label{fig:00_template}
\end{figure*}

% Figure 01: Figure 01
\clearpage
\begin{figure*}[p]
    \pdfbookmark[2]{Figure 01}{figure_id_01}
    \centering
    \includegraphics[width=0.95\textwidth]{./src/figures/caption_and_media/jpg/Figure_ID_01_workflow.jpg}
    \caption{
\textbf{
Figure 01
}
\smallskip
\\
Description for figure 01.
}
    \label{fig:01_workflow}
\end{figure*}

\n

%% ----------------------------------------
%% END of DOCUMENT
%% ----------------------------------------
\end{document}

