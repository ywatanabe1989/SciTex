\documentclass{article}
\usepackage{amsmath}
\usepackage{graphicx}

\title{Sample LaTeX Document for Testing}
\author{SciTex Test Suite}
\date{\today}

\begin{document}

\maketitle

\section{Introduction}

This is a sample LaTeX document used for testing the SciTex system. The document contains various elements that are typically found in scientific manuscripts, such as equations, references, and figures.

\subsection{Background}

Scientific writing requires precise language and consistent terminology. Previous work has shown that structured documents with proper formatting enhance readability and comprehension.

\section{Methods}

\subsection{Data Collection}

Data was collected using standard protocols. The experimental setup consisted of multiple trials under controlled conditions.

\subsection{Analysis}

Statistical analysis was performed using standard methods. The significance level was set at $p < 0.05$.

\begin{equation}
f(x) = \frac{1}{\sigma\sqrt{2\pi}} e^{-\frac{1}{2}\left(\frac{x-\mu}{\sigma}\right)^2}
\end{equation}

\section{Results}

Our experiments showed consistent results across all trials. Figure~\ref{fig:sample} shows the primary outcomes.

\begin{figure}
\centering
\rule{8cm}{6cm} % This is a placeholder for an actual figure
\caption{Sample figure showing experimental results.}
\label{fig:sample}
\end{figure}

\section{Discussion}

The results demonstrate the effectiveness of our approach. These findings are consistent with previous research in the field.

\section{Conclusion}

In conclusion, we have presented a sample document that can be used for testing the SciTex system.

\end{document}